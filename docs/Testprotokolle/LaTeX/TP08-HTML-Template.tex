%!TEX root = Testprotokolle.tex

\section{HTML-Template}

\renewcommand{\arraystretch}{1.5}

\begin{center}
 \begin{tabular}{|p{0.21\textwidth}|p{0.7\textwidth}|}
	\hline
	\textbf{Identifikator}  & /T0080/ \\
	\hline
	\textbf{Datum} & \today \\
	\hline
	\textbf{Bearbeiter} & Michael Hertel \\
	\hline
	\textbf{Bezeichnung} & HTML-Template \\
	\hline
	\textbf{Ausgangszustand} & 
		Vorhandenes Repository mit Beispielrechnung und vorhandenes HTML-Template. \\
	\hline
	\textbf{Aktionen} & 
		Rechnung auf Basis des HTML-Templates generieren (mit \textit{inv render}). \\
	\hline
	\textbf{Zielzustand} & 
		HTML-Datei, in der die Platzhalter durch die Parameter der Rechnung ersetzt oder bei nicht vorhanden Werten entfernt wurden. \\
	\hline
	\textbf{Status} & Test erfolgreich \\
	\hline
 \end{tabular}
\end{center}
