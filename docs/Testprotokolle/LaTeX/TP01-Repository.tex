%!TEX root = Testprotokolle.tex

\section{Repository}

\renewcommand{\arraystretch}{1.5}

\begin{center}
 \begin{tabular}{|p{0.21\textwidth}|p{0.7\textwidth}|}
	\hline
	\textbf{Identifikator}  & /T0010/ \\
	\hline
	\textbf{Datum} & \today \\
	\hline
	\textbf{Bearbeiter} & Michael Hertel \\
	\hline
	\textbf{Bezeichnung} & Repository \\
	\hline
	\textbf{Ausgangszustand} & 
		Verzeichnis außerhalb eines bestehenden Repositories. \\
	\hline
	\textbf{Aktionen} & 
		Anlegen eines neuen Rechnungs-Repositorys mit \textit{inv init}. \\
	\hline
	\textbf{Zielzustand} & 
		Valides Rechnungs-Repository im Zielverzeichnis. \\
	\hline
	\textbf{Status} & Test erfolgreich \\
	\hline
 \end{tabular}
\end{center}
