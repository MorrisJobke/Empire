%!TEX root = Pflichtenheft.tex

\section{Testszenarien und Testf\"alle}



\begin{description}
  \item[/T0010/]
    \textit{Erstellen einer Rechnung mit einem HTML-Template:}
   Es wird zun\"achst ein g\"ultiges HTML-Template erstellt. Dieses wird dann mit zuvor in das Rechnungssystem eigegebenen Testdaten gef\"ullt. Das dadurch entstandene Rechnungs-PDF wird zuletzt auf Vollst\"andigkeit und Korrektheit \"uberpr\"uft.
	\begin{itemize}
   		\item Erstellen einer Rechnung mit einem HTML-Template -> vollständige Rechnung als PDF
		\item Erstellen einer Rechnung mit einem Latex-Template -> vollständige Rechnung als PDF
		\item Testen des Programms auf Reaktion auf ein defektes/unvollständiges Template -> Fehlermeldung
		\item Berechnen der Endjahresbilanz / Gewinn-Verlust-Überschussrechnung ->korrekte Ausgabe
		\item Anlegen mehrerer Rechnungs-Repositorys -> mehrere Repositorys mit jeweils einer DB
   \end{itemize}
\end{description}


