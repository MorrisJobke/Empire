%!TEX root = Pflichtenheft.tex

\section{Zielbestimmungen}



\subsection{Musskriterien}

\begin{itemize}
	\item Das Programm stellt ein Rechnungssystem für Kleinunternehmer zur Verfügung.
	\item Die Rechnungen können per Plugin im HTML- und PDF-Format ausgegeben werden.
	\item Der Nutzer kann durch von ihm vorgegebene Latex oder HTML-Templates die Rechnungen seinen Wünschen anpassen.
	\item Die Verwaltung der Rechungen basiert auf Rechnungs-Repositorys, welche vom Programm erstellt und verwaltet werden können.
	\item Durch die Vererbung von Attributen können Wiederholungen von Eingaben von häufig auftretenden Rechnungsdaten eingespart werden.
	\item Dem Benutzer wird das Auffinden von Rechnungen durch eine Suchfunktion erleichtert.
	%\item Backend, Abstraktion

\end{itemize}

\subsection{Wunschkriterien}

\begin{itemize}
	\item Die ausgegebenen Rechnungen werden digital signiert.
	\item Dem Nutzer werden die Endjahresbilanzen und Gewinn-Verlust-Überschussrechnungen bereitgestellt.
	\item Es wird eine GUI zur leichteren Bedienung bereitgestellt.
	\item Die eindeutige Rechnungsnummer kann frei gestaltet werden, sodass nicht nur eine fortlaufende Durchnummerierung zur Verfügung steht.
\end{itemize}

\subsection{Abgrenzungskriterien}

\begin{itemize}
	\item Das Rechnungssystem beachtet keine Steuern.
	\item Das Programm ist nicht für Großunternehmen geeignet.
\end{itemize}
