%!TEX root = Pflichtenheft.tex

\section{Testszenarien und Testfälle}



\begin{description}
  \item[/T0010/]
    \textit{Erstellen einer Rechnung mit einem HTML-Template:}
   Es wird zunächst ein gültiges HTML-Template erstellt. Dieses wird dann mit zuvor in das Rechnungssystem eigegebenen Testdaten gefüllt. Das dadurch entstandene Rechnungs-PDF wird zuletzt auf Vollständigkeit und Korrektheit überprüft.
	\begin{itemize}
   		\item Erstellen einer Rechnung mit einem HTML-Template -> vollständige HTMl-Datei
		\item Erstellen einer Rechnung mit einem Latex-Template -> vollständiges Latex-Datei (z.B. alle Platzhalter ersetzt)
		\item Testen des Programms auf Reaktion auf ein defektes/unvollständiges Template -> Fehlermeldung
		\item korrektes Anlegen des Rechnungs-Repositorys 
		\item CRUD von Rechnungen
		\item Rechnungsnummer-Syntax -> korrekte Rechnungsnummer
   \end{itemize}
\end{description}

Eingangszustand - Ausgangszustand


