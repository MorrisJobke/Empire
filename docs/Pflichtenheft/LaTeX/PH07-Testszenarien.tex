%!TEX root = Pflichtenheft.tex

\section{Testszenarien und Testfälle}



\begin{description}
  \item[/T0010/]
	\textit{Repository}
	\begin{description}
		\item[Ausgangszustand]
		Verzeichnis außerhalb eines bestehenden Repositories.
		\item[Aktion]
		Anlegen eines neuen Rechnungs-Repositorys mit \textit{inv init}.
		\item[Erwarteter Zielzustand]
		Valides Rechnungs Repository im Zielverzeichnis.
	\end{description}

  \item[/T0020/]
	\textit{Erstellen}
	\begin{description}
		\item[Ausgangszustand]
		Vorhandenes Rechnungs-Repository.
		\item[Aktion]
		Erstellen einer neuen Rechnung mit \textit{inv create}.
		\item[Erwarteter Zielzustand]
		Erstellte Rechnung im Rechnungs-Repository.
	\end{description}

  \item[/T0030/]
	\textit{Auslesen}
	\begin{description}
		\item[Ausgangszustand]
		Vorhandenes Rechnungs-Repository mit vorhandener Rechnung.
		\item[Aktion]
		Auslesen einer Rechnung mit \textit{inv show}.
		\item[Erwarteter Zielzustand]
		Ausgabe der auszulesenden Rechnung.
	\end{description}

  \item[/T0040/]
	\textit{Bearbeiten}
	\begin{description}
		\item[Ausgangszustand]
		Vorhandenes Rechnungs-Repository mit vorhandener Rechnung.
		\item[Aktion]
		Verändern der Daten der Rechnung \textit{inv modify}.
		\item[Erwarteter Zielzustand]
		Aktualisierung der Daten im Repository.
	\end{description}

  \item[/T0050/]
	\textit{Löschen}
	\begin{description}
		\item[Ausgangszustand]
		Vorhandenes Rechnungs-Repository mit vorhandener Rechnung.
		\item[Aktion]
		Löschen der Rechnung im Repository \textit{inv remove}.
		\item[Erwarteter Zielzustand]
		Rechnung nicht mehr im Repository vorhanden.
	\end{description}

  \item[/T0060/]
	\textit{Rechnungsnummer-Syntax}
	\begin{description}
		\item[Ausgangszustand]
		Rechnungsnummer-Syntax mit enthaltenem Platzhalter für Jahr, Monat, Tag und den Teil der fortlaufenden Nummer.
		\item[Aktion]
		Generieren mehrere Rechnungen mit sich in Jahr, Monat und Tag unterscheidenden Datums.
		\item[Erwarteter Zielzustand]
		Die Jahr, Monat und Tag Platzhalter entsprechen dem angegeben Datum und die fortlaufende Nummer erhöht sich bei weiteren neugenerierten Rechnung immer nur dann, wenn es schon Rechnungen mit identischen Platzhalter Werten gibt. Dabei erhöht sich die fortlaufende Nummer um die Anzahl der bereits mit diesen Werten der Platzhalter generierten Rechnungen plus eins.
	\end{description}

  \item[/T0070/]
	\textit{Latex-Template}
	\begin{description}
		\item[Ausgangszustand]
		Vorhandenes Repostory mit Beispielrechnung und vorhandenes Latex-Template.
		\item[Aktion]
		Rechnung auf Basis des Latex-Templates generieren (mit \textit{inv render}).
		\item[Erwarteter Zielzustand]
		Latex Datei in der die Platzhalter durch die Parameter der Rechnung ersetzt oder bei nicht vorhanden Werten entfernt wurden.
	\end{description}

  \item[/T0080/]
	\textit{HTML-Template}
	\begin{description}
		\item[Ausgangszustand]
		Vorhandenes Repostory mit Beispielrechnung und vorhandenes HTML-Template.
		\item[Aktion]
		Rechnung auf Basis des HTML-Templates generieren (mit \textit{inv render}).
		\item[Erwarteter Zielzustand]
		HTML Datei in der die Platzhalter durch die Parameter der Rechnung ersetzt oder bei nicht vorhanden Werten entfernt wurden.
	\end{description}
\end{description}
