%!TEX root = Handbuch.tex

\lstset{
  literate= {Ö}{{\"O}}1 {Ä}{{\"A}}1 {Ü}{{\"U}}1 {ß}{{\ss}}1 {ü}{{\"u}}1
 {ä}{{\"a}}1 {ö}{{\"o}}1
 }

\section{Ausführliches Anwendungsbeispiel}
Im Folgenden soll ein Überblick über die Möglichkeiten dieses Programmes gegeben werden. Dazu nutzen wir das als Vorlage enthaltene Template und erstellen drei Rechnungen an zwei Kunden mit jeweils unterschiedlichen Posten.

Zu Beginn wechseln wir in das Verzeichnis unserer Wahl und erstellen dort ein Repository:
\begin{lstlisting}[style=Bash]
$ mkdir Rechnungen
$ cd Rechnungen
$ emp init
\end{lstlisting}

Um in diesem Beispiel eine bessere Übersicht zu gewährleisten, kopieren wir außerdem das Template in den Ordner des Repositories. Im allgemeinen Fall ist es aber gleichgültig, wo sich das Template befindet.
\begin{lstlisting}[style=Bash]
$ cp <Pfad des Templates> ./invoice.htm
\end{lstlisting}


Wir definieren im Wurzelverzeichnis des Repositories unsere persönlichen Daten, da diese in allen Rechnungen gleich sein sollen. Dazu zählen beim vordefinierten Template die Anschrift, die Bankverbindung, unsere Kontaktdaten und eine Funktion, die den Preis der Posten sowie den Gesamtbetrag berechnen wird:

\begin{lstlisting}[style=Bash]
$ emp add from_name "Max Mustermann"
$ emp add from_street "Musterstraße 42"
$ emp add from_city "94837 Musterstadt"
$ emp add from_mobile 0111/1234567
$ emp add from_email max.mustermann@example.org
$ emp add from_ustidnr DE999999999
$ emp add acct_name Musterbank
$ emp add acct_nr 123456
$ emp add acct_blz 65432100
$ emp add total "return price * quantity; return lhs + rhs;"
\end{lstlisting}

Die letzte Zeile in obrigen Listing definiert einen Funktions-Wert. Die erste Funktion berechnet später den Preis pro Posten und die zweite die Summe dieser - somit den Gesamtbetrag.\vspace{0.5em}

Dasselbe Ergebnis kann man übrigens auch mit dem \textbf{iadd}-Befehl erreichen, auf welchen wir aber erst später zurückgreifen werden.
\vspace{0.5em}

Im Folgenden definieren wir die Variablen, welche die Grundlage für die im Template enthaltene Collection bilden. Um herauszufinden welche das sind bemühen wir den \textbf{status}-Befehl und übergeben das Template als Parameter:
\begin{lstlisting}[style=Bash]
$ emp status ../../invoice.htm
\end{lstlisting}

Besonders interessant ist dabei der Abschnitt der vom Template genutzten, aber im Repository nicht enthaltenen Collections. Dort wird uns die Collection \grqq item\grqq mit den folgenden Werten angezeigt:

\begin{lstlisting}[style=Bash]
Collections:
============
Used by template and not defined(1):
	items, containing elements (got from Template):
		description  price        quantity
		total        unit
\end{lstlisting}

Die Deklaration erfolgt mit dem \textbf{create}-Befehl:
\begin{lstlisting}[style=Bash]
$ emp create description string
$ emp create price double
$ emp create quantity double
$ emp create unit string
\end{lstlisting}

Nun kann die Collection deklariert werden:

\begin{lstlisting}[style=Bash]
$ emp cadd items description price quantity unit
\end{lstlisting}

Im Anschluss erstellen wir für jeden Kunden einen Ordner und hinterlegen in diesem die Daten des Kunden:
\begin{lstlisting}[style=Bash]
$ mkdir Kunde1 Kunde2
$ cd Kunde1
$ emp add to_name "Muster Co."
$ emp add to_street "Musterallee 42"
$ emp add to_city "56346 Musterhausen"
\end{lstlisting}

Die Bennenung der Ordner spielt hier keine Rolle (solange keine Collection mit dem gewünschten Namen existiert), sollte aber zur Förderung der Übersichtlichkeit sinnvoll gewählt werden.

Für den 2. Kunden nutzen wir nun den iadd-Befehl:
\begin{lstlisting}[style=Bash]
$ cd ../Kunde2
$ emp iadd ../invoice.htm
\end{lstlisting}

Es erfolgt eine interaktive Abfrage der im Template genutzten, aber im Repository noch nicht gesetzten Werte. Wir wollen nur die Kundendaten eintragen und überspringen die restlichen Abfragen mit der Enter-Taste.

\vspace{0.5em}
\begin{tabular}{| l | l |}
	\hline
	to\_city & 85423 Weißlauf \\
	\hline
	to\_name & Peter Schlegel \\
	\hline
	to\_street & Münchener Straße 133 \\
	\hline
\end{tabular}
\vspace{0.5em}

Jetzt können wir unseren Zwischenstand mit dem \textbf{status}-Befehl überprüfen. Dazu nutzen wir folgendes Kommando:
\begin{lstlisting}[style=Bash]
$ emp status ../invoice.htm
\end{lstlisting}

Wir erhalten eine Auflistung aller bereits gesetzten Variablen und Collections. Zusätzlich werden die weiteren benötigen Variablen angezeigt. Falls weitere Werte existieren, die nicht im Template benötigt werden, werden diese ebenfalls in einer gesonderten Kategorie angezeigt.

\vspace{0.5em}
Nun erstellen wir eine Rechnung für Kunde 1. Dazu erstellen wir im Verzeichnis von Kunde 1 einen Ordner und füllen ihn mit den restlichen Informationen:

\begin{lstlisting}[style=Bash]
$ cd ../Kunde1
$ mkdir Auftrag1
$ cd Auftrag1
$ emp iadd ../../invoice.htm
\end{lstlisting}

\begin{tabular}{| l | l |}
	\hline
	accomplishment & 05.01.2012 \\
	\hline
	date & 09.01.2012 \\
	\hline
	nr & 2012-13 \\
	\hline
\end{tabular}
\vspace{0.5em}

Der Collection sollten noch ein paar Zeilen hinzugefügt werden:
\begin{lstlisting}[style=Bash]
$ emp cfill items  description:Webseite price:99.99 quantity:1
$ emp cfill items  description:Softwareentwicklung price:25.00 unit:Stunden quantity:5
$ emp cfill items  description:Datenrettung price:250.00 quantity:1
\end{lstlisting}

Vor dem Rendern überprüfen wir nochmals unseren aktuellen Status:
\begin{lstlisting}[style=Bash]
$ emp status ../../invoice.htm
\end{lstlisting}
Alle Werte sollten nun grün oder türkis dargestellt werden. Grün symbolisiert, dass die Variablen im aktuellen Ordner erstellt wurden, türkise Variablen befinden sich hingegen in einem übergeordneten Verzeichnis.

Das Rendern erfolgt mit dem \textbf{render}-Befehl:
\begin{lstlisting}[style=Bash]
$ emp render ../../invoice.htm
\end{lstlisting}

Die erste Rechnung ist damit fertig. Die Erstellung der restlichen zwei Rechnungen erfolgt größtenteils analog zur Ersten. Um aber ein paar weitere Möglichkeiten aufzuzeigen werden diese trotzdem erläutert.
Zunächst erstellen wir ein Verzeichnis für die zweite Rechnung.

\begin{lstlisting}[style=Bash]
$ cd ..
$ mkdir Auftrag2
$ cd Auftrag2
\end{lstlisting}

Es folgt eine Demonstration, wie man einen übergeordneten Wert(in einem Elternverzeichnis) überschreiben kann. Hierzu wird angenommen, dass die Rechnung an eine andere Adresse geschickt werden soll. Wir erstellen dazu die Werte im aktuellen Verzeichnis neu:

\begin{lstlisting}[style=Bash]
$ emp add to_street "Musterweg 3"
$ emp add to_city "52246 Musterdorf"
\end{lstlisting}

Mit dem \textbf{status}-Befehl lässt sich leicht erkennen, dass sich \textbf{to\_street} und \textbf{to\_city} im aktuellen Verzeichnis befinden und unsere gerade zugewiesenen Werte enthalten.
Das Property \textbf{to\_name} hat hingegen seinen Wert beibehalten, da es in einem der Elternverzeichnisse vorgefunden wurde. Gleiches gilt für all unsere eigenen Daten im Pfad Rechnungen.

\vspace{0.5em}
Im Folgenden müssen wir wieder die Collection mit Daten füllen. Die Definition dafür ist bereits von der ersten Rechnung vorhanden und wir können direkt mit dem Füllen der Collection beginnen, dazu nutzen wir die interaktive Variante von \textbf{cfill}:
\begin{lstlisting}[style=Bash]
$ emp cfill items
\end{lstlisting}
Wenn genug Zeilen eingegeben wurden, bestätigt man einfach mit der Enter-Taste.

Das Rendern erfolgt wieder mit dem bereits bekannten Aufruf:
\begin{lstlisting}[style=Bash]
$ emp render ../../invoice.htm
\end{lstlisting}
