%!TEX root = Handbuch.tex

\section{Glossar}

\begin{description}
  \item[Lua] \hfill \\
  Lua ist eine imperative Skriptsprache. In Empire werden Funktions-Werte werden als Lua interpretiert und ausgeführt.
  \item[Property] \hfill \\
  Ein Property stellt eine Eigenschaft eines Objektes dar. In Empire sind Properties die kleinstmöglichen und somit unteilbaren Einheiten.
  \item[Repository] \hfill \\
  Repositories bilden die übergeordnete Einheit zu den Properties. Sie dienen als Behälter für diverse Properties für eine spezielle Aufgabe.
  \item[Collection] \hfill \\
  Collections ermöglichen das Sammeln von Properties zu einem spezifischen Bereich. Sie sind wie Properties einem Repository untergeordnet und bestehen aus einer Liste von Properties.
  \item[Map-/Reduce-Funktion] \hfill \\
  Die Map-/ Reduce-Funktion ist eine Funktion, welche aus einer Map- und einer Reduce-Funktion besteht. Eine Map-Funktion führt eine übergebene Operation auf eine Menge von Elementen aus. Dabei wird jedes Element einzeln mittels der übergebenen Funktion ausgewertet. Eine Reduce-Funktion führt eine Reduktions-Operation über eine Menge von Elementen aus. Die übergebene Funktion wird dazu auf alle Elemente nacheinander angewendet, um diese miteinander zu einem Element zu verknüpfen. Bei einer Map-/ Reduce-Funktion wird zunächst die Map-Funktion auf eine Menge von Elementen ausgeführt und im Anschluss eine Reduktion der Elemente durchgeführt.
\end{description}
