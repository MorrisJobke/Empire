%!TEX root = Handbuch.tex

\section{Backup}

Bei der Behandlung seriöser Datenmengen wie z.B. der von Rechnungen, ist es enorm wichtig eine
Redundanz mit Hilfe von Backups zu garantieren. Weiterhin müssen Backups systemweit konsistent funktionieren.
Diesen Anforderungen wird das Speichermodell des Programms dadurch gerecht, indem alle relevanten Daten bei jeder
Aktion sofort auf die Festplatte geschrieben werden. Dadurch ist es möglich ein Rechnungsrepository einfach auf
ein anderes Speichermedium zu kopieren.


%Für das Backup der Nutzerdaten ist der Benutzer des Programms selbst zuständig.
%Um eine Sicherung anzulegen müssen lediglich sämtliche erstellte Daten manuell kopiert werden.
%Dazu kann beispielsweise das copy-Kommando der Unix-Shell benutzt werden:
%\begin{lstlisting}[style=Bash]
%$ cp -pR source/path destination/path
%\end{lstlisting}
