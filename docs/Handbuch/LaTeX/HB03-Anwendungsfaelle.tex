%!TEX root = Handbuch.tex

\lstset{
  literate= {Ö}{{\"O}}1 {Ä}{{\"A}}1 {Ü}{{\"U}}1 {ß}{{\ss}}1 {ü}{{\"u}}1
 {ä}{{\"a}}1 {ö}{{\"o}}1
 }

\section{Erklärung der wesentlichen Anwendungsfälle}

\subsection{Initialisieren eines Repositories}
Das Initialisieren eines Repositories geschieht mit dem folgenden Befehl:
\begin{lstlisting}[style=Bash]
$ emp init
\end{lstlisting}
Zu beachten ist, dass ein Repository nicht erzeugt werden kann, wenn im vorhandenen oder einem übergeordneten Verzeichnis bereits ein Repository existiert.

\subsection{Hinzufügen von Werten zu einem Repository}
Werte können mit dem add-Befehl hinzugefügt werden:
\begin{lstlisting}[style=Bash]
$ emp add <Schlüssel> [<Typ>] <Wert>
\end{lstlisting}
Die Angabe von einem Datentyp ist optional. Wenn keiner angegeben wird, erfolgt eine automatische Typ-Bestimmung.
Es folgen einige Beispiele für die Typ-Erkennung:

\begin{tabular}{| l | l |}
	\hline
	int & 1234, 123 \\
	\hline
	float & 1234.0, .123, -.123, +0.123 \\
	\hline
	string & s1234, 12s3, \grqq Text mit Leerzeichen\grqq , Adresse \\
	\hline
\end{tabular} 

Wenn eine automatische Erkennung nicht gewünscht ist, muss ein Typ angegeben werden. Im Folgenden dazu ein Beispiel:
\begin{lstlisting}[style=Bash]
$ emp add floatString string 1.2314
\end{lstlisting}

\subsection{Löschen von bereits hinzugefügten Werten}
Das Löschen von in einem Repository enthaltenen Werten erfolgt mit dem Befehl remove:
\begin{lstlisting}[style=Bash]
$ emp remove [Optionen] <Schlüssel>
\end{lstlisting}

Dabei wird jedoch lediglich ein Wert gelöscht (der des aktuellen Verzeichnisses). Die Werte in anderen Verzeichnissen und auch die Typ-Definition bleiben erhalten.

Um auch die Typ-Definition entfernen zu können nutzen Sie die Option --all:
\begin{lstlisting}[style=Bash]
$ emp remove --all <Schlüssel>
\end{lstlisting}
Zu beachten ist allerdings, dass mit diesem Befehl alle vorhandenen Werte mit diesem Schlüssel und auch deren Definition unwiederuflich entfernt werden.

\subsection{Modifizieren von bereits hinzugefügten Werten}
\subsection{Erstellen eines Templates}
\subsection{Rendern eines Templates}
Mit dem folgenden Befehl wird ein Template gerendert:
\begin{lstlisting}[style=Bash]
$ emp render <Template-Datei> <Ausgabe-Datei>
\end{lstlisting}

\subsection{Anzeigen von bereits hinzugefügten Werten}
Um einen Überblick über den aktuellen Status zu erhalten, kann das show-Kommando genutzt werden:
\begin{lstlisting}[style=Bash]
$ emp show [<Template-Datei>]
\end{lstlisting}
Wenn hierbei kein Parameter übergeben wird, erhalten Sie eine Auflistung Ihrer bereits verwendeten Werte und verfügbaren Wert-Definitionen.
Bei Übergabe einer Template-Datei als Parameter werden die im Template vorhandenen und noch nicht definierten Werte aufgelistet. Außerdem erhalten Sie eine Übersicht über die bereits definierten und im Template benötigten Werte.
