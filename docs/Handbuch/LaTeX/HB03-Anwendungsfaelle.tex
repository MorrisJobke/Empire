%!TEX root = Handbuch.tex

\lstset{
  literate= {Ö}{{\"O}}1 {Ä}{{\"A}}1 {Ü}{{\"U}}1 {ß}{{\ss}}1 {ü}{{\"u}}1
 {ä}{{\"a}}1 {ö}{{\"o}}1
 }

\section{Erklärung der wesentlichen Anwendungsfälle}

\subsection{Initialisieren eines Repositories}
Das Initialisieren eines Repositories geschieht mit dem folgenden Befehl:
\begin{lstlisting}[style=Bash]
$ emp init
\end{lstlisting}
Zu beachten ist, dass ein Repository nicht erzeugt werden kann, wenn im vorhandenen oder einem übergeordneten Verzeichnis bereits ein Repository existiert.

\subsection{Hinzufügen von Werten zu einem Repository}
Werte können mit dem add-Befehl hinzugefügt werden:
\begin{lstlisting}[style=Bash]
$ emp add <Schlüssel> [<Typ>] <Wert>
\end{lstlisting}
Die Angabe eines Datentyps ist optional. Wenn keiner angegeben wird, erfolgt eine automatische Typ-Bestimmung.
Es folgen einige Beispiele für die Typ-Erkennung:

\vspace{0.5em}

\begin{tabular}{| l | l |}
	\hline
	int & 1234, 123 \\
	\hline
	float & 1234.0, .123, -.123, +0.123 \\
	\hline
	string & s1234, 12s3, \grqq Text mit Leerzeichen\grqq , Adresse \\
	\hline
	FunctionType & \grqq return price * quantity; return lhs + rhs;\grqq\\
	\hline
\end{tabular}

\vspace{0.5em}

Wenn eine automatische Erkennung nicht gewünscht ist, muss ein Typ angegeben werden. Im Folgenden dazu ein Beispiel:
\begin{lstlisting}[style=Bash]
$ emp add floatString string 1.2314
\end{lstlisting}

\subsection{Besonderheit: FunctionType Eigenschaft}
Um sich wiederholende Berechnungen zu vereinfachen, gibt es einen speziellen Funktions-Typ, der einfache Lua-Anweisungen auswerten kann. Er nimmt genau 2 solcher Anweisungen auf, eine um z.B. einzelne Posten einer Rechnung zusammen zu rechnen (vgl. \grqq Map-Funktion\grqq) und eine, die wiederrum diese berechneten Werte zu einem einzelnen Endergebnis zusammenführt (vgl. \grqq Reduce-Funktion\grqq).
\\\\
Die Variablen innerhalb der ersten dieser beiden Funktionen können Namen von allen verfügbaren Werten (auch in Collections enthaltenen) annehmen. Die zweite Funktion benutzt nur die beiden Variablen \grqq lhs \grqq und \grqq rhs \grqq , welche für den linken und den rechten Operanden des Ausdrucks stehen.
\\\\
Unterstützt werden zur Zeit jedoch nur einfachste Ausdrücke der Form \grqq return \emph{ausdruck}; return \emph{ausdruck}; \grqq. (Siehe obere Tabelle)


\subsection{Löschen von bereits hinzugefügten Werten}
Das Löschen von in einem Repository enthaltenen Werten erfolgt mit dem Befehl remove:
\begin{lstlisting}[style=Bash]
$ emp remove [Optionen] <Schlüssel>
\end{lstlisting}

Dabei wird jedoch lediglich ein Wert gelöscht (der des aktuellen Verzeichnisses). Die Werte in anderen Verzeichnissen und auch die Typ-Definition bleiben erhalten.

Um auch die Typ-Definition entfernen zu können nutzen Sie die Option --all:
\begin{lstlisting}[style=Bash]
$ emp remove --all <Schlüssel>
\end{lstlisting}
Zu beachten ist allerdings, dass mit diesem Befehl alle vorhandenen Werte mit diesem Schlüssel und auch deren Definition unwiederuflich entfernt werden.

\subsection{Modifizieren von bereits hinzugefügten Werten}
Das Ändern eines im Repository enthaltenen Wertes erfolgt mit dem Befehl modify:
\begin{lstlisting}[style=Bash]
$ emp modify <Schlüssel> <Wert>
\end{lstlisting}
Dabei wird der zu dem Schlüssel gehörende Wert durch den im Kommando angegeben Wert ersetzt.

\subsection{Erstellen eines Templates}
Ein Template kann mit jedem beliebigen Texteditor erstellt werden. Dabei gilt folgender Syntax für die Platzhalter, welche beim Rendern durch die entsprechenden Werte ersetzt werden.
\\\\
Ein Platzhalter wird mit dem @-Zeichen eingeleitet. Alle darauf folgenden alphanumerischen Werte und Unterstriche ergeben den Schlüssel eines im Repository enthaltenen Wertes.
\vspace{0.5em}
\begin{lstlisting}[style=Bash]
Anschrift:
@Vorname @Nachname
@AnschriftZeile1
@AnschriftZeile2
@PLZ @Ort
\end{lstlisting}

Zusätzlich können Tabellen mit Hilfe von Collections abgebildet werden. Dazu muss auf das @-Zeichen und den darauf folgenden Schlüssel eine geöffnete geschweifte Klammer folgen.
\vspace{0.5em}
\begin{lstlisting}[style=Bash]
| Anzahl  | Posten   |
@Einkaufsliste{
| @Anzahl | @Posten  |
}
\end{lstlisting}

Dabei ist darauf zu achten, dass die geöffnete geschweifte Klammer direkt auf den Schlüssel folgt. Mit einer geschloßenen geschweiften Klammer wird daraufhin die Darstellung der Collection abgeschloßen.

Um das @-Zeichen selbst darzustellen, kann dieses mit Hilfe eines weiteren @-Zeichens escaped werden.

\begin{lstlisting}[style=Bash]
Email: name.vorname@@domain.tld
\end{lstlisting}

\subsection{Rendern eines Templates}
Mit dem folgenden Befehl wird ein Template gerendert:
\begin{lstlisting}[style=Bash]
$ emp render <Template-Datei> [<Ausgabe-Datei>] [-f]
\end{lstlisting}

Die beiden letzteren Parameter sind optional. Falls der Parameter für die Ausgabe-Datei nicht gesetzt ist, wird der Dateiname des Templates benutzt und die Datei im aktuellen Verzeichnis abgelegt. Mit der Option -f kann man das Überschreiben erzwingen, falls die Ausgabe-Datei bereits vorhanden ist. Diese Option wird an jeder Stelle in der Parameter-Liste erkannt.

\subsection{Anzeigen von bereits hinzugefügten Werten}
Um einen Überblick über den aktuellen Status zu erhalten, kann das status-Kommando genutzt werden:
\begin{lstlisting}[style=Bash]
$ emp status [<Template-Datei>]
\end{lstlisting}
Wenn hierbei kein Parameter übergeben wird, erhalten Sie eine Auflistung Ihrer bereits verwendeten Werte und verfügbaren Wert-Definitionen.
Bei Übergabe einer Template-Datei als Parameter werden die im Template vorhandenen und noch nicht definierten Werte aufgelistet. Außerdem erhalten Sie eine Übersicht über die bereits definierten und im Template benötigten Werte.
