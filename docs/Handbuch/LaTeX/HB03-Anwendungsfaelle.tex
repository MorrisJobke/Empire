%!TEX root = Handbuch.tex

\section{Erklärung der wesentlichen Anwendungsfälle}

\subsection{Initialisieren eines Repositorys}
Das initialisieren eines Repositorys geschieht mit dem folgenden Befehl:
\begin{verbatim}
emp init
\end{verbatim}
Zu beachten ist, dass ein Repository nicht erzeugt werden kann, wenn im vorhandenen oder einem übergeordnerten Verzeichnis bereits ein Repository existiert.

\subsection{Hinzufügen von Werten zu einem Repository}
Werte können mit diesem Befehl hinzugefügt werden:
\begin{verbatim}
emp add <Schlüssel> [<Typ>] <Wert>
\end{verbatim}
Die Angabe von einem type ist optional. Wenn keiner angegeben wird erfolgt eine automatische Typ-Bestimmung.
Hier einige Beispiele für die Typ-Erkennung:

\begin{tabular}{| l | l |}
	\hline
	int & 1234, 123 \\
	\hline
	float & 1234.0, .123, -.123, +0.123 \\
	\hline
	string & s1234, 12s3, \"Text mit Leerzeichen\", Adresse \\
	\hline
\end{tabular} 

Wenn eine solche Erkennung nicht gewünscht ist, dann muss ein Typ angegeben werden. Im folgenden dazu ein Beispiel:
\begin{verbatim}
emp add floatString string 1.2314
\end{verbatim}

\subsection{Löschen von bereits hinzugefügten Werten}
\subsection{Modifizieren von bereits hinzugefügten Werten}
\subsection{Erstellen eines Templates}
\subsection{Rendern eines Templates}
\subsection{Anzeigen von bereits hinzugefügten Werten}