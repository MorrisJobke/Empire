%!TEX root = Handbuch.tex

\section{Zielgruppe}

Das Design der Ihnen hier vorliegenden Anwendung ist sehr modular und anpassungsfreundlich konzipiert. Damit zielt die Software auf Nutzer ab, welche
es Vorziehen ihre Anwendungsumgebung auf die von Ihnen preferierten Umstände anzupassen und sie vor allen Dingen unter Kontrolle zu haben.
\\\\
Die konsolen-orientierte Bedienung des Programms setzt dabei eine gewisse Kentniss voraus, welche aber für z.B. freiberufliche Entwickler, Administratoren,
Software-Designer oder ehrenamtliche Vereins-Buchhalter, welche sich eine kostenaufwendige und spezielle Lösung dann ersparen könnten, keine Hürde darstellen sollte.
\\\\
Der Nutzerkreis wird zudem durch das Prinzip, welchem Empire zu Grunde liegt, noch erweitert, da es wesentlich mehr Einsatzmöglichkeiten gibt, als es als Rechnungssystem
einzusetzen.
\\\\
Es wäre z.B. denkbar, dass ein Software-Entwickler ein Repository zum Speichern von Konfigurationsdateien für sein Programm verwendet.
Dabei könnten z.B. die Collections als Konfigurations-Sektionen benutzt werden. Weiter könnte ein Kleinunternehmer Serien-Briefe für seine Kunden schreiben.

%Die Zielgruppe dieser Anwendung sind in erster Linie freiberufliche Entwickler und Designer mit einer Vorliebe für Konsolenanwendungen. Empire ermöglicht es diesen Personen, mit ihnen größtenteils bereits bekannten Befehlen, ihre Rechnungen zu erstellen und zu verwalten.
%
%\parskip 12pt
%
%Da sich die Anwendungsfälle jedoch nicht nur auf Rechnungen beschränken, gehören auch Nutzer, die keine freiberuflichen Entwickler oder Designer sind, zur Zielgruppe, wenn sie eine Vorliebe für Konsolen und eine Verwendung für eine einfache Key-Value Datenverwaltung mit angebundener Template-Engine haben.
