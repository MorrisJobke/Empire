%!TEX root = Handbuch.tex

\section{Kurzbeschreibung}

Empire ist eine unter der GPL lizensierte Konsolenanwendung, die es dem Nutzer ermöglicht Schlüsselwerte in einem Repository zu organisieren.
Als Repository wird dabei eine Ordnerstruktur bezeichnet, in welcher sich die Schlüsselwerte befinden.
Mittels der definierten Werte kann aus einem vorher erstellten Template ein Dokument gerendert werden. Ein Template stellt dabei eine einfache
Textdatei dar, in der Schlüssel in einer entsprechenden Syntax eingefügt werden. Dabei beschränken sich die Schlüssel-Wert-Paare nicht nur auf einfache
Typen wie Zeichenketten oder Zahlen, sondern ermöglichen auch das Gruppieren von Werten durch Collections und die Verarbeitung von Funktionen.
\\\\
Durch den allgemeinen Ansatz der Software ist eine hohe Anzahl an Einsatzmöglichkeiten denkbar. In dieser Anleitung wird dabei im Wesentlichen
auf die Anwendung als Rechnungssystem eingegangen. Es ist aber auch denkbar ein durch Empire erstelltes Repository als Speicher für Konfigurationen
oder als Adressbuch zu benutzen.
\\\\
Um den allgemeinen und minimalen Ansatz des Programms fortzuführen, werden alle Daten direkt in das jeweilige Dateisystem geschrieben.
Das ermöglicht ein unkompliziertes Weiterverarbeiten der Daten durch beispielsweise Scripts, Backuptools und andere Anwendungen.
