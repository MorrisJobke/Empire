%!TEX root = Handbuch.tex

\section{Installationsanleitung}

\subsection{Vorraussetzungen für die Installation}
Bevor Empire kompiliert werden kann, müssen folgende Pakete installiert werden:

\begin{itemize}
	\item cmake
	\item Boost-Bibliothek
	\item Lua-Bindings für C++
	\item PCRE-Bibliothek
\end{itemize}

Dies entspricht folgenden Paketen:

\underline{Ubuntu:}
\begin{lstlisting}[style=Bash]
$ sudo apt-get install cmake libboost-test-dev libluabind-dev libpcre3-dev
\end{lstlisting}

\underline{ArchLinux:}
\begin{lstlisting}[style=Bash]
$ sudo pacman -S boost cmake lua pcre
\end{lstlisting}

\underline{Mac:}
\begin{lstlisting}[style=Bash]
$ brew install boost cmake lua
\end{lstlisting}

\subsection{Kompilieren des Programmes}

Im Anschluss beziehen wir den Quellcode von der Projekt-Seite. Danach wechseln wir in das \textbf{build}-Verzeichnis und kompilieren.

\begin{lstlisting}[style=Bash]
$ git clone git://github.com/brainafk/Empire.git
$ cd Empire/build
$ cmake ../src
$ make
\end{lstlisting}

\subsection{Installieren des Programmes}

Zum installieren wird dann folgender Befehl benutzt.

\begin{lstlisting}[style=Bash]
$ sudo make install
\end{lstlisting}

\emph{Achtung: Bitte beachten sie, dass das Make-Programm bereits bestehende Dateien überschreibt.}
